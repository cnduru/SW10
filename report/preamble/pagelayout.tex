%[BEGIN] Chapter customisation
  \usepackage{graphics}
  \usepackage{titlesec}
  \titleformat{\chapter}[display]
    {\normalfont\Large\raggedleft}
    {\MakeUppercase{\chaptertitlename}%
      \rlap{\resizebox{!}{1.5cm}{~\thechapter~} \rule{5cm}{1.5cm}}}
    {10pt}{\Huge\bf}
  \titlespacing*{\chapter}{0pt}{30pt}{20pt}
%[END]

%[BEGIN] Bibliography in TOC
  \usepackage[nottoc]{tocbibind} % Works only with standard classes
%[END]

%[BEGIN] Margin adjustment
  \usepackage[total={5.7in,8.75in},top=1.5in,left=1in,bottom=1.5in]{geometry}
  %\usepackage[a4paper]{geometry}
%[END]

%[BEGIN] define whitespace placement
  \raggedbottom
%[END]

% Table setup
\renewcommand{\arraystretch}{1.4}

% Numbering
\renewcommand{\thesection}{\thechapter.\arabic{section}}
\renewcommand{\thesubsection}{\thesection.\arabic{subsection}}
\renewcommand{\thesubsubsection}{\thesubsection.\arabic{subsubsection}}
\renewcommand{\thefigure}{F\arabic{chapter}-\arabic{figure}}
\renewcommand{\thetable}{T\arabic{chapter}-\arabic{table}}

%[BEGIN] Define page headers and footers - must be done AFTER any manipulation of margins and the like
 
  % Document visual header definition
  %\setlength{\headheight}{15pt}
 	
 	\pagestyle{fancy}
 	\renewcommand{\chaptermark}[1]{\markboth{\MakeUppercase{#1}}{}}
%	\renewcommand{\chaptermark}[1]{\markboth{\small{\scshape{#1}}\ \scshape{\small{\thechapter}}}{}}
%	\renewcommand{\sectionmark}[1]{\markright{\scshape{\small{\thesection}}\ \small{\scshape{#1}}}}

	\fancyhf{}
	\fancyhead[LO,LE]{\rightmark}%{\MakeUppercase{Section} \thesection}
	\fancyhead[RO,RE]{\leftmark}
  \fancyfoot[LE,RO]{\thepage}
  
  \renewcommand{\footnoterule}{%
    \kern 0pt
    \hrule width 0.5\textwidth height 0.5pt
    \kern 4pt
  }
  
  %    E for even page
  %    O for odd page
  %    L for left side
  %    C for centered
  %    R for right side
  %    
  %    \thepage adds number of the current page.
  %    \thechapter adds number of the current chapter.
  %    \thesection adds number of the current section.
  %    \chaptername adds the word "Chapter" in English or its equivalent in the current language.
  %    \leftmark adds name and number of the current top-level structure (for example, Chapter for reports and books classes; Section for articles ) in uppercase letters.
  %    \rightmark adds name and number of the current next to top-level structure (Section for reports and books; Subsection for articles) in uppercase letters.
	
%[END]