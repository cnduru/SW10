% definition of example environment
%\newcommand\getcurrentref[1]{%
% \ifnumequal{\value{#1}}{0}
%  {??}
%  {\the\value{#1}}%
%}

%\newcommand\exref[1]{%
%  Example \hyperref[#1]{\getcurrentref{chapter}.\ref{#1}}%
%}

%\newcommand\resetexamplecounter{%
 % \ifnum\value{chapter}>\value{prevchapter}
 %   \setcounter{examplecounter}{0}
 % \fi
%}
%\newcounter{prevchapter}
%\newcounter{examplecounter}
%\newenvironment{example}[1]{
%    \resetexamplecounter
%    \setcounter{prevchapter}{\value{chapter}}
%    \refstepcounter{examplecounter}%
%  \paragraph{Example \getcurrentref{chapter}.\arabic{examplecounter}: #1}%
%  \quad
%}
\usepackage{ifthen}
\newtheorem{innerExample}{Example}
\crefname{innerExample}{Example}{Examples}
\numberwithin{innerExample}{chapter}
\usepackage{mdframed}
\newmdenv[
  topline=false,
  bottomline=false,
  rightline=false,
  skipabove=\topsep,
  skipbelow=\topsep,
  leftmargin=-20pt,
  rightmargin=-10pt,
  innertopmargin=0pt,
  innerbottommargin=0pt
]{siderules}

\DeclareDocumentEnvironment{example}{O {} m}
    {
      \vspace{1em}
        
      %\vspace{-2em}
      \ifthenelse{\equal{#1}{}}{\begin{innerExample}#2}{\begin{innerExample}[#1]#2}
      \end{innerExample} \mbox{}
      \vspace{-2em}
      \begin{siderules}
    }{\end{siderules}}

% Part descriptions
% Usage \newpart{part name}{part text}
\newcommand{\newpart}[2]{
	\part[#1]{#1\\
		\begin{minipage}[c]{10cm}
		\centering
		\vspace{2cm}
		\normalsize{\textnormal{
			\textit{#2}
		}}
		\end{minipage}
	}
}

%\lstset{language=C,
%    basicstyle=\ttfamily,
%    keywordstyle=\bfseries,
%    showstringspaces=false,
%    morekeywords={include, printf, CFP, process, flag, start, counter, protection, verify}
%}

\usepackage{listings}
\usepackage{color}

\definecolor{dkgreen}{rgb}{0,0.6,0}
\definecolor{gray}{rgb}{0.5,0.5,0.5}
\definecolor{mauve}{rgb}{0.58,0,0.82}

\lstset{frame=tb,
  language=Java,
  aboveskip=3mm,
  belowskip=3mm,
  showstringspaces=false,
  columns=flexible,
  basicstyle={\small\ttfamily},
  numbers=none,
  numberstyle=\tiny\color{gray},
  keywordstyle=\color{blue},
  commentstyle=\color{dkgreen},
  stringstyle=\color{mauve},
  breaklines=true,
  breakatwhitespace=true,
  tabsize=3
}

\newcommand{\semsp}{\hspace{20px}}
\newcommand{\semnl}{\vspace{10px}}
\newcommand{\jcl}{TinyJCL\xspace}
\newcommand{\jc}{Java Card\xspace}
% Insert figure
\newcommand{\insertfigure}[4]{%
    \begin{figure}[H] %
        \center %
        \includegraphics[#1]{figures/#2} %
        \caption{#3} %
        \label{#4} %
        \vspace{-5pt}
    \end{figure} %
}

% Insert figure
\newcommand{\inserttexfigure}[3]{%
    \begin{figure}[H] %
        \center %
        \input{figures/#1} %
        \caption{#2} %
        \label{#3} %
        \vspace{-5pt}
    \end{figure} %
}

% Allow line break in texttt
\newcommand*\justify{%
  \fontdimen2\font=0.4em% interword space
  \fontdimen3\font=0.2em% interword stretch
  \fontdimen4\font=0.1em% interword shrink
  \fontdimen7\font=0.1em% extra space
  \hyphenchar\font=`\-% allowing hyphenation
}

\let\oldtexttt\texttt
\renewcommand*{\texttt}[1]{\oldtexttt{\justify #1}}

% Name ref to a section including its section numbering
\newcommand{\namedref}[1]{%
  \nameref{#1} (\cref{#1})\xspace
}

% Add Blankpage command
\newcommand{\Blankpage}{ %
    \newpage               %
    \thispagestyle{empty}  %
    \mbox{}                %
    \newpage               %
}

% TODO NOTES COMMANDS
\newcommand{\namedtodo}[5]
{
  \stepcounter{todocounter}
  \ifthenelse{\equal{#1}{inline}}
  % INLINE TODO
  {
    \todo[color=#4,caption={\textbf{\inserttodocounter #3: } #2},inline]
    {\textbf{\inserttodocounter #3: }\color{#5}#2}
  }
  % NORMAL TODO
  {
    \todo[color=#4,caption={\textbf{\inserttodocounter #3: } #2}]
    {\textbf{\inserttodocounter #3: }\color{#5}#2}
  }
}
% Counter format
\newcommand{\inserttodocounter}{\#\thetodocounter\;}

% FORMAT FOR TODO:
% Missing color
\definecolor{BitterSweet}{rgb}{1.0, 0.44, 0.37}
% [inline] (optional) besked navn bg-farve font-farve
\newcommand{\kristian}[2][]{\namedtodo{#1}{#2}{Kristian}{NavyBlue!35}{black}}
\newcommand{\kri}[1]{\kristian[inline]{#1}}
\newcommand{\christoffer}[2][]{\namedtodo{#1}{#2}{Christoffer}{LimeGreen}{black}}
\newcommand{\ch}[1]{\christoffer[inline]{#1}}
\newcommand{\erik}[2][]{\namedtodo{#1}{#2}{Erik}{Apricot!90}{black}}
\newcommand{\dennis}[2][]{\namedtodo{#1}{#2}{Dennis}{BitterSweet!90}{black}}
\newcommand{\den}[1]{\dennis[inline]{#1}}
\newcommand{\style}[2][]{\namedtodo{#1}{#2}{Style}{GreenYellow}{black}}
\newcommand{\finished}[2][]{\namedtodo{inline}{#2}{Finished}{GreenYellow}{black}}

%Projct commands
\newcommand{\projectsubject}{\large{Modelling Java Bytecode}}
\newcommand{\projecttitle}{\Large Assessing Bit Flip Attacks and Countermeasures}

% Project classes
%\newcommand{\RssFeed}{\lstinline{RssFeed}\xspace}

% Tokenize
\newcommand{\createtoken}[1]{\framebox{%
\strut #1}}

\makeatletter
\def\tokenize#1{%
\expandafter\tokenize@i#1 \@nil}
\def\tokenize@i#1 #2\@nil{%
  \createtoken{#1}
  \ifx\relax#2\relax%
  % is empty
  \else
  \tokenize@i#2\@nil
  \fi}
\makeatother

% Create a itemize-like line of text in a table
\newcommand{\tabitem}{~~\llap{\textbullet}~~}

% Commonly used words/terms
%\newcommand{\rails}{Ruby on Rails\xspace}

% Abbreviations \newacronym{label}{abbreviation}{full description} % USAGE: \gls{label}
%\newacronym{voip}{VoIP}{Voice over Internet Protocol}
%\newacronym{osm}{OSM}{OpenStreetMap}
%\newacronym{p2p}{P2P}{Peer-to-peer}
%\newacronym{isp}{ISP}{Internet Service Provider}
%\newacronym{sip}{SIP}{Session Initiation Protocol}
%\newacronym{xmpp}{XMPP}{Extensible Messaging and Presence Protocol}
%\newacronym{xml}{XML}{Extensible Markup Language}
%\newacronym{iq}{IQ}{Info/Query}
%\newacronym{gcm}{GCM}{Google Cloud Messaging}
%\newacronym{rtp}{RTP}{Real-time Transport Protocol}
