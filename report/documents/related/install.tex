\subsection{\texttt{install()}}\label{subsec:jcinstall}
The \texttt{install} method creates an instance of the \texttt{Applet} subclass~\cite[p. 65]{java_card_spec}. Depending on the application of the card, this method is called \textit{once} in a card's life time, from either the manufacturer's or card distributor's side. Examples of such cards are credit cards and SIM cards, since it could pose a security risk to allow other applets than those intended to be on the card, to be installed.\\\\
The install method should perform all necessary initialisations and must perform a call to the \texttt{register} method. If the call to \texttt{register} is not performed successfully, or an exception is thrown before the call, the installation is not considered successful. If the installation fails, the Java Card runtime environment performs the necessary clean up actions when control is returned to it. After a successful installation, the applet can be selected with the \texttt{select APDU} command. APDUs are described in~\cref{subsec:apdu}.
