\subsection{Smart Card}\label{subsec:smartcard}\ch{cites needed?}
A smart card is a small embedded circuit which can process and store small amounts of information. The card does not have a power supply and does not work without a terminal to supply it with power. A well-known use of smart cards is to embed them in plastic cards and use them as e.g. credit cards. They have three types of memory: RAM, ROM and EEPROM. Information such as variables can be stored in RAM and altered, but disappear after the chip is powered down. ROM memory cannot be altered and persists across power-ups and downs. On a \jc this is used to store the \jc Virtual Machine. The EEPROM is persistant across power-ups and down but information stored in it can be altered and on a \jc it is usually used to store third party applets and information used by the virtual machine.\\\\
The programs running on a \jc are called applets and a card can have multiple applets installed. They communicate with a terminal through Application Protocol Data Unit packets, described in \cref{subsec:apdu}. A credit card applet can for example inform a credit card terminal that an entered pin code is incorrect by sending a packet. This will allow the terminal to block a money transaction before it occurs, if an incorrect pin code is entered.