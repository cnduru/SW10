\section{Program Analysis with Sawja}
% intro about Sawja
\ch{should we mention stackless presentations JBir and A3Bir?}
We use the tool \textit{Sawja} (Static Analysis Workshop for JAva)\cite{sawja}, which relies on \textit{Javalib}, to extract information from Java class files. The tool can analyse Java code\ch{should we write Java or Java Byte code?} and create a call graph, which we then use for rewriting purposes, described in\ch{ref}.\\\\

\cref{javaorig}

% line numbers

% inlining of constant pool



\begin{lstlisting}[caption=Java sample.,language=Java,label=lst:javaorig]
public void foo(boolean b){
    (b ? new A() : new B()).bar();
}
}
\end{lstlisting}

\begin{lstlisting}[caption=Sawja sample.,language=Java,label=lst:javasawja]
public void foo ( bool 1 ) ;
		Concrete Method
    	Not parsed

0.  iload 1
1.  ifeq 13
4.  new A
7.  dup
8.  invokespecial void A.<init> ( )
        A.<init>
11. goto 10
14. new B
17. dup
18. invokespecial void B.<init> ( )
        B.<init>
21. invokevirtual short A.bar ( )
        B.bar
        A.bar
24. pop
25. return

\end{lstlisting}