The time at which a bit flip is performed, matters in terms of how much of the system can be affected, e.g. at run-time versus when a card is powered off. Logically, a greater attack surface equals a greater probability of affecting a piece of memory that will cause a desirable outcome for an attacker. An example could be a method which is only runs once compared to a method which might be called ten times. Assuming methods of similar size and memory usage, the second method would have a probability ten times the that of the method which is only called once, to be hit by a bit flip. It should be noted that even if a method has a greater probability of being hit by a bit flip, it is not guaranteed that there is a greater chance that the bit flip will bring the card into a situation compromising security. This depends on the nature of the method.\\\\
% install()
We have decided to not perform any simulation of bit flip attacks in the \texttt{install} method, described in~\cref{sec:jc}, and protection against these. The reason for this is, that for cards like credit cards, the method is only executed once and the execution happens at the manufacturer or distributor of the card.\ch{move to future works?}

%
