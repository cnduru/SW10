\chapter{Introduction}
% bite here
In May 2011 Economic Interest Group noticed that smart cards stolen in France were being used in Belgium~\cite{fun}. It turned out a group of criminals, aided by an engineer, managed to perform a man-in-the middle attack on the credit cards by placing a chip on top of the original chip. The attack bypassed PIN verification by intercepting communication between a credit card terminal and the original chip. As a consequence, a transaction would be approved whether the correct PIN was entered on the terminal or not. The gang is estimated to have caused damages below \EUR{600,000}. They sold items purchased with the stolen cards on the black market, and managed to exploit over 7,000 transactions before being apprehended. Smart cards are found in many places today, everywhere from phone SIM cards in phones, access cards and credit cards. The first wide spread use was French pay phone cards in 1983~\cite[p. 366]{modbank}.\\

% purpose of the report
\noindent As criminals are finding ever more sophisticated ways to exploit computer systems we rely on, on a daily basis, it is important to always stay ahead.
This report aims to investigate the security risks of fault injections aimed at \jc\ch{mention that we have performed experiments on Java code and not Java card, but the bit flips are modelled after Java card}, specifically attacks relying on random bit flips. 
In this process we have built a tool for constructing UPPAAL~\cite{upptut} models from Java bytecode which allows us to simulate random bit flip occurrences. 
The foundation of the model building is based on our previous work~\cite{javasec} with the \jcl language, found in~\cref{sec:semintro}.\\\\
In this report we will take a look at three selected attack countermeasures; code duplication, call graph integrity and control flow integrity which are expected to improve the security of \jc bytecode. 
These countermeasures will be applied to Java Bytecode as part of our evaluation.
