\subsection{Control Flow Integrity} 
As described by~\cite{agl}, is a technique which allows a programmer to annotate code to denote sensitive regions. The programmer then inserts flag variables which increment a counter, as shown in \cref{lst:examplecfisnip1}. It is also possible to insert verification points where the counter is checked to verify the flag value, shown in \cref{lst:examplecfisnip}. The idea is that if a change in the call graph occurs, e.g. a method call is skipped because of a fault in the program counter, the flag will have the wrong value. A full implementation on the code sample can be seen in \cref{lst:examplecfi}.

\begin{lstlisting}[caption={Java code example of the control flow integrity countermeasure incrementing the control flow flag.},numbers=none,label={lst:examplecfisnip1}]
0.  GETSTATIC 3
3.  PUSH 1
4.  ADD
5.  PUTSTATIC 3
...
\end{lstlisting}

\begin{lstlisting}[caption={Java code example of the control flow integrity countermeasure checking the control flow flag.},numbers=none,label={lst:examplecfisnip}]
...
 4.  LOAD  0
 5.  INVOKESPECIAL 32   // processVerifyPIN()
 8.  GETSTATIC  2
 9.  PUSH 5
13.  IF_CMPEQ 25
<bit flip detected code>
25.  RETURN
\end{lstlisting}


