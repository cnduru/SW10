%!TEX root = ../../master.tex
\section{Fault Semantics}
\label{sec:semFault}
We now introduce a fault model  formalising the fault injections which can happen. This serves the purpose of showing exactly how a certain fault affects the Java Card, whether it be a program flow change or a change in the memory.\\

To describe the fault semantics, we represent $Values$ and $Addresses$ as bit strings. 
We then use a Hamming distance of one, $\equiv_1$, to represent a bit flip, so $a \equiv_1 b$ means $a$ differs from $b$ with only one bit. 
It can also be stated as 
$$\exists x \in \mathbb{N}: a \equiv_1 b \Rightarrow a = b \oplus 2^x$$
where $\oplus$ is the binary \texttt{XOR} operation. 
It can be expressed similarly with sequences $$A \equiv_1 B \Rightarrow e \neq e' | A = (x_0, \ldots, x_{n-1}, e, x_{n+1} \ldots,
 x_m) \land B = (x_0, \ldots,x_{n-1}, e', x_{n+1} \ldots, x_m)$$ where $n$ is the position of the differentiating element in the sequences $A$ and $B$.


\subsection{DATAFAULT}
A data fault can occur three places: the operand stack, the local variables or the heap. These faults are formalised in the \texttt{DF\_OPS}, \texttt{DF\_LOC} and \texttt{DF\_HEAP} rules respectively.

$$\inference[DF\_OPS]{  
ops = (x_0,  \ldots, x_n, \ldots, x_m) \semsp
ops' = (x_0, \ldots, v, \ldots, x_m) \semnl \\
v \equiv_1 x_n \semsp
0 \leq n \leq m 
}
{CP, P \vdash \langle H, (CS, \langle  mid, loc, ops, pc \rangle)\rangle \Rightarrow \langle H, (CS, \langle mid, loc, ops', pc \rangle)\rangle}$$

$$\inference[DF\_LOC]{  
fid \in loc \semsp
 v = loc(fid) \semsp
  v' \equiv_1 v
}
{CP, P \vdash \langle H, (CS, \langle  mid, loc, ops, pc \rangle)\rangle  \Rightarrow \langle H, (CS, \langle mid, loc[fid \mapsto v'], ops, pc \rangle)\rangle }$$


$$\inference[DF\_HEAP]{  
a \in H \semsp
 v = H(a) \semsp
 v' \equiv_1 v
}
{CP, P \vdash \langle H, CS\rangle  \Rightarrow \langle H[a \mapsto v'], CS\rangle }$$


\subsection{PROGRAMFLOWFAULT} \label{sec:pff}
This fault causes a change in the program flow of the applet. There are two cases: In the first case, either the fault changes the program counter, which has the consequence of changing which instruction is to be executed next. But since we have defined the program counter to only span locally within the method body, \texttt{PFF\_PC} only describes a change in program flow within the method body. In the second case, \texttt{PFF\_M} describes a fault which changes the method identifier, \textit{mid}, of the method to be executed. The fault described by \texttt{PFF\_M} will change the program flow to another method, outside of the current stack frame, but at the same program counter value within the new method's stack frame.

$$\inference[PFF\_PC]{  
 \semsp pc' \equiv_1 pc
}
{CP, P \vdash \langle H, (CS, \langle  mid, loc, ops, pc \rangle)\rangle  \Rightarrow \langle H, (CS, \langle mid, loc, ops, pc' \rangle)\rangle}$$

$$\inference[PFF\_M]{  
mid' \equiv_1 mid
}
{CP, P \vdash \langle H, (CS, \langle  mid, loc, ops, pc \rangle)\rangle  \Rightarrow \langle H, (CS, \langle mid', loc, ops, pc \rangle)\rangle }$$

\subsection{INSTRUCTIONFAULT}
This fault causes a change in an instruction, e.g. changing a \texttt{ADD} to a \texttt{POP} by changing a single bit in the instruction.

$$\inference[INST\_F]{
inst(P,mid, pc) = I \semsp
P \equiv_1 P' \semnl \\
I,I' \in Instructions \semsp
I' \neq I \semsp
inst(P',mid, pc) = I'
}
{\langle CP, P, H, (CS, \langle mid, loc, ops, pc \rangle) \rangle \Rightarrow \langle CP, P', H, (CS, \langle mid, loc, ops, pc \rangle)\rangle }$$
%%% Local Variables:
%%% mode: latex
%%% TeX-master: "../../master"
%%% End:

