%!TEX root = ../../master.tex
\subsection{INVOKE\_VIRTUAL}
\label{app:invokevirtual}
\texttt{INVOKEVIRTUAL} is similar to \texttt{INVOKESTATIC} but in addition an object reference from the operand stack is stored as the first local variable, and the method for the actual class is resolved by a method lookup, inspired by~\cite{dalvik}. For this we introduce two functions $signa$, and $methodLookup$. $signa = MethodID \to Signature$, where $Signature$ is the method's signature e.g. name and parameters. And $methodLookup$ used to lookup the intended method identifier, either from the class itself or a super class, defined as: \\\\
$methodLookup(mid, cl) = $ \vspace{-10px}
\[
\begin{cases}
  \perp & if\ cl = \perp \\
  mid'  & if\ mid' \in cl.Methods \wedge signa(mid') = signa (mid)  \\
  methodLookup(mid, cl.Class) & otherwise
\end{cases}
\]


$$\inference[INVOKEVIRTUAL]{
inst(P,mid,pc) = \texttt{INVOKEVIRTUAL }mid' \semsp 
CP(mid') = pn \semnl \\
ops = (x_0, \ldots, x_n, objr, p_1, \ldots, p_{pn}) \semsp
ops' = (x_0, \ldots, x_{n}) \semnl\\ 
methodLookup(H(objr).Class, mid') = mid'' \semsp
mid'' \neq \perp \semnl\\
loc' = [0 \mapsto objr, 1 \mapsto p_{1}, \ldots, pn \mapsto p_{pn}]
}
{CP, P \vdash \langle H, (CS, \langle mid, loc, ops, pc \rangle)\rangle \Rightarrow }$$
$$\hspace{130px}\langle H, (CS, \langle mid, loc, ops', pc \rangle, \langle mid'', loc', \epsilon, 0 \rangle)\rangle$$
%%% Local Variables:
%%% mode: latex
%%% TeX-master: "../../master"
%%% End:
