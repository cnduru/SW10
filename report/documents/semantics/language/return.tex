\subsection{RETURN}
\texttt{RETURN} is used when returning from a method.
The result of a \texttt{RETURN} depends on the state of the operand stack when called.
If the operand stack is not empty the top element will be the return value.

$$\inference[RETURN]{
inst(P,mid',pc') = \texttt{RETURN} \semsp
ops = (x_0, \ldots, x_n) \semnl \\
ops' \neq \epsilon \semsp
ops' = (x'_0, \ldots, x'_n) \semsp
ops'' = (x_0, \ldots, x_n, x'_n)} 
{CP, P \vdash \langle H, (CS, \langle mid, loc, ops, pc \rangle, \langle mid', loc', ops', pc' \rangle)\rangle \Rightarrow}$$
$$ \langle H, (CS, \langle mid, loc, ops'', pc + 1 \rangle)\rangle$$

In following case, where the operand stack is empty, it will return without adding an element to the previous frame's operand stack.

$$\inference[RETURN VOID]{
inst(P,mid',pc') = \texttt{RETURN} \semsp
ops' = \epsilon }
{CP, P \vdash \langle H, (CS, \langle mid, loc, ops, pc \rangle, \langle mid', loc', ops', pc' \rangle)\rangle \Rightarrow}$$
$$ \langle H, (CS, \langle mid, loc, ops, pc + 1 \rangle)\rangle$$
%%% Local Variables:
%%% mode: latex
%%% TeX-master: "../../master"
%%% End:
