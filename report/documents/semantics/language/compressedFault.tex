\subsection{Fault Models}
We use fault models to formally represent how fault injections affect our modelled execution. With the \jcl semantics we also defined a number of fault models to describe how various bit flips can affect the execution, a summery can be seen in \cref{tab:fault} and the full definition can be seen in \cref{sec:semFault}. Each rule defines how a single bit flip will affect the program state or configuration, for multiple bit flips a rule must be applied multiple times. However for the purpose of this report we will only focus on single bit flip.


\begin{table}[H]
\centering
\label{tab:fault}
\begin{tabular}{p{.35\textwidth}|p{.55\textwidth}}
\textbf{Rule} 					  & \textbf{Description} \\ \hline
DATA\_FAULT             		  & Describes a change in the operand stack, local variables or the heap, caused by a bit flip.\\ \hline 
PROGRAM\_FLOW\_FAULT			  & Describes a change in the program flow or method identifier due to a bit flip. \\ \hline
INSTRUCTION\_FAULT				  & Describes a change from one instruction to another, caused by a bit flip. \\ \hline
\end{tabular}
\caption{Fault semantics with a short description. Note that stack refers to the operand stack.}
\end{table}