\subsection{\jcl}
\ch{what do we do with this table?}
\begin{table}[H]
\centering
\label{tab:instr}
\begin{tabular}{p{.25\textwidth}|p{.75\textwidth}}
\textbf{Rule}			  & \textbf{Description} \\ \hline
NOP             		  & No operation. Only increments the program counter. \\ \hline
PUSH \textit{v}           & Pushes parameter \textit{v} on top of the stack. \\ \hline
POP             		  & Removes and discards top element of the stack.\\ \hline
ADD             		  & Adds the two top elements of the stack and pushes the result back onto the stack.  \\ \hline
DUP             		  & Duplicates the top element of the stack and pushes it onto the stack. \\ \hline
GOTO \textit{a}           & Jumps to a certain address in the program.  \\ \hline
IF\_CMPEQ \textit{a}   	  & Compares the two top stack elements and performs a conditional jump to \textit{a}  \\ \hline
INVOKE\_STATIC     		  & Calls a static method.  \\ \hline
INVOKE\_VIRTUAL 		  & ...  \\ \hline
INVOKE\_SPECIAL			  & ...  \\ \hline
RETURN                    & Returns from a method. If the stack is non-empty the top value is returned.   \\ \hline
PUT\_STATIC \textit{fid}  & Writes the top value of the stack to a class variable on the heap.  \\ \hline
GET\_STATIC \textit{fid}  & Pushes a class variable from the heap onto the stack. \\ \hline
LOAD \textit{i}           & Loads a local variable onto the stack.  \\ \hline
STORE \textit{i}          & Stores a value from the stack in a local variable.\\ \hline
PUT\_FIELD \textit{fid}   & Stores a value from the top of the stack and stores it in a field in an object.   \\ \hline
GET\_FIELD \textit{fid}	  & Reads a field in an object and pushes it onto the stack. \\ \hline
NEW   		              & Creates an object on the heap and pushes a reference to it onto the stack.  \\ \hline  
\end{tabular}
\caption{\jcl semantics with a short description. Note that stack refers to the operand stack.}
\end{table}