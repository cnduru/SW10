\section{\jcl}
\jcl is a small language based on Java Card bytecode. Just as Java Card \jcl is a stack based language with a heap. \jcl was created and formalised in \cite{javasec}, and the semantics can be seen in \cref{sec:semintro}
It supports the many of features of Java bytecode such as classes and method invocation which is showcased in \Cref{app:sample1}. 
A brief overview of the instructions can be seen in \cref{tab:instr}.\\

\noindent Program errors are not resented in \jcl, it is assumed that if no rule is legal the virtual machine will exit, e.g. if the current instruction is resolved to be \texttt{LOAD } $i$ and the parameter $i$ is not part of the domain for the $Local$ function for the top elements of the call stack, the rule would not be legal. \\\\
\jcl has been expanded upon in this report. 
The original semantics did not capture the notion of private method calls and exceptions. 
In the following \jcl is expanded with rules for \texttt{INVOKESPECIAL} and \texttt{THROW}, to support invocation of private methods and exceptions, respectively.

\begin{table}
\centering
\begin{tabular}{p{.30\textwidth}|p{.63\textwidth}}
\label{tab:instr}
\textbf{Rule / Instruction}			  & \textbf{Description} \\ \hline
NOP             		  & No operation. Only increments the program counter. \\ \hline
PUSH \textit{v}           & Pushes parameter \textit{v} on top of the stack. \\ \hline
POP             		  & Removes and discards top element of the stack.\\ \hline
ADD             		  & Adds the two top elements of the stack and pushes the result back onto the stack.  \\ \hline
DUP             		  & Duplicates the top element of the stack and pushes it onto the stack. \\ \hline
GOTO \textit{a}           & Jumps to a certain address in the program.  \\ \hline
IF\_CMPEQ \textit{a}   	  & Compares the two top stack elements and performs a conditional jump to \textit{a}  \\ \hline
INVOKESTATIC \textit{mid}    		  & Calls a static method.  \\ \hline
INVOKEVIRTUAL \textit{mid}		  & Calls a virtual method.  \\ \hline
RETURN                    & Returns from a method. If the stack is non-empty the top value is returned.   \\ \hline
PUTSTATIC \textit{fid}  & Writes the top value of the stack to a class variable on the heap.  \\ \hline
GETSTATIC \textit{fid}  & Pushes a class variable from the heap onto the stack. \\ \hline
LOAD \textit{i}           & Loads a local variable onto the stack.  \\ \hline
STORE \textit{i}          & Stores a value from the stack in a local variable.\\ \hline
PUTFIELD \textit{fid}   & Stores a value from the top of the stack and stores it in a field in an object.   \\ \hline
GETFIELD \textit{fid}	  & Reads a field in an object and pushes it onto the stack. \\ \hline
NEW   		              & Creates an object on the heap and pushes a reference to it onto the stack.  \\ \hline  
\end{tabular}
\caption{\jcl semantics with a short description. Note that stack refers to the operand stack.}
\end{table}
\subsection{Expanding \jcl}
\jcl created and formalised by \cite{javasec}, seen in \cref{sec:semintro}, has been expanded upon in this report. The original semantics\ch{streamline to either "operational semantics" or "semantics" across all of the report} did not describe private method calls and exceptions. In the following, these additions are made.

\subsubsection{\texttt{INVOKESPECIAL}}
$$\inference[INVOKESPECIAL]{
inst(P,mid,pc) = \texttt{INVOKESPECIAL }mid' \semsp 
CP(mid') = pn \semnl \\
ops = (x_0, \ldots, x_n, objr, p_1, \ldots, p_{pn}) \semsp
ops' = (x_0, \ldots, x_{n}) \semnl\\
loc' = [0 \mapsto objr, 1 \mapsto p_{1}, \ldots, pn \mapsto p_{pn}])}
{CP, P \vdash \langle H, (CS, \langle mid, loc, ops, pc \rangle)\rangle \Rightarrow }$$
$$\langle H, (CS, \langle mid, loc, ops', pc \rangle, \langle mid', loc', \epsilon, 0 \rangle)\rangle$$


\subsubsection{\texttt{Exceptions}}
$CS = (CS', \langle mid, loc, ops, pc \rangle)$ \\ 
$ops = (x_0, \ldots, x_n, objr)$\vspace{5px} \\
$exceptionLookup(P, CS) = $ \vspace{-10px} \\
\[
\begin{cases}
  (CS', \langle mid, loc, (objr), pc' \rangle  & if \hspace{5px} exceptDef(P, mid, pc, objr) = pc' \neq \perp\\
  exceptionLookup(P, CS') & otherwise
\end{cases}
\]

$$\inference[THROW]{
inst(P,mid,pc) = \texttt{THROW} }
{CP, P \vdash \langle H, CS \rangle \Rightarrow \langle H, exceptionLookup(P,CS)\rangle}$$

