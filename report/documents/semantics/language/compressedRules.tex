\section{\jcl}
\begin{table}[]
\centering
\label{tab:instr}
\begin{tabular}{l|l}
\hline
NOP             		  & No operation. Only increments the program counter. \\ \hline
PUSH \textit{v}           & Pushes parameter \textit{v on top of the stack.} \\ \hline
POP             		  & Removes and discards top element of the stack.\\ \hline
ADD             		  & Adds the two top elements of the stack and pushes the result back onto the stack.  \\ \hline
DUP             		  & Duplicates the top element of the stack and pushes it onto the stack. \\ \hline
GOTO \textit{a}           & Jumps to a certain address in the program.  \\ \hline
IF\_CMPEQ \textit{a}   	  & Compares the two top stack elements and performs a conditional jump to \textit{a}  \\ \hline
INVOKE\_STATIC     		  &   \\ \hline
RETURN          &   \\ \hline
PUT\_STATIC     &   \\ \hline
GET\_STATIC     &   \\ \hline
LOAD            &   \\ \hline
STORE           &   \\ \hline
INVOKE\_VIRTUAL &   \\ \hline
PUT\_FIELD      &   \\ \hline
GET\_FIELD      &   \\ \hline
NEW             &   \\ \hline  
\end{tabular}
\caption{\jcl semantics with a short description. Note that stack refers to the operand stack.}
\end{table}