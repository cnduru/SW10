\chapter{Conclusion}
In this report an overview of the \jc platform as well as attacks and countermeasures, aimed at the platform is presented. The primary focus is on bit flips, and how they can affect static and running behaviour of programs.\\\\
% %
Our previous work~\cite{javasec} argued that it is difficult to measure and compare the effectiveness of countermeasures. We have tried to offer a solution to this problem by providing an automated approach for converting Java bytecode to UPPAAL SMC models, which allows us to test properties of a model, in its query language. We present an approach to model all major features of the Java bytecode language in UPPAAL. The logic in our model conversion is based on \jcl semantics, which we have expanded upon with additional rules.\\\\
% %
We have based our conversion on the Java static analysis tool \textit{Sawja}'s native Java bytecode representation. In extension, we have investigated its call graph generation functionality, which is useful for automatic implementation of control flow based countermeasures.\\\\
% %
%Experiments
In our experiments, we used our tool to construct models based on two code samples, and four fault models. Three fault injection countermeasures were also modelled for each code sample, and UPPAAL queries were then used to assess various properties of the models which has given us a basis for comparison of the countermeasures.

\section{Future Work}

%inst fault, %Javacard inst bit encoding
%parameter fault etc

%multi path in program

\ch{mention how instruction fault could be modelled and why we haven't included it in our experiments (we didn't want to model the entire instruction set)}
\subsection{Sawja's Call Graph Representation}
The tool Sawja, described in \cref{sec:sawja} can be used to automate the call graph integrity countermeasure. An analysis of the call graph shows which methods are callable, which should have unique ids and which should be grouped as is the case in \cref{lst:javaorig}. Since it is not possible to know which of the \texttt{bar} methods are called, the caller id which they check when called from \texttt{foo} are the same.\\\\
In addition, the control flow integrity countermeasure can also be automated by analysing the call graph. It enables the flag counter to be correctly incremented through a call graph analysis, which will then be used to determine the flag values to check before and after each method call.

\ch{mention that an extension to fault model could improve measurements}