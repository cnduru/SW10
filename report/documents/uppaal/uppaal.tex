\chapter{UPPAAL and Formal Verification}
\section{Property Verification}
UPPALL has its own query language used to verify properties of a model\cite[p. 7]{upptut}. The language is a simplified version of timed computation tree logic. UPPAAL's query language consists of \textit{state formulae} and \textit{path formulae}. The path formulae can be categorised into three categories: reachability, safety and liveness.

\paragraph{State formulae}
%deadlock special case
\paragraph{Reachability properties}
Reachability properties express the notion that a property, $\varphi$, can \textit{possibly} be satisfied on some path, going from the initial location of the model. In UPPAAL it is expressed as \texttt{E<>$\varphi$}. This could for example be used to verify whether a variable \texttt{i} in the model, along some path going from the initial location will have the value $2$ by querying the model with \texttt{E<>i == 2}.

A special 
\paragraph{Safety formulae}
\paragraph{Liveness formulae}

\section{SMC}
extra edges for waiting