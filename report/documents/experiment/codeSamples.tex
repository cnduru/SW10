\label{chap:samples}
\begin{lstlisting}[caption={Mocked Java example code from the Java Card samples},label={lst:example}]
// Example.java
public class Example 
{
    public static void main(String[] args) {
        try
        {
		  Example hw = new Example();
        }catch (Exception ex)
        {

        }
    }

    public Example() throws Exception
    {
        processVerifyPIN();
    }

    private void processVerifyPIN() throws Exception
    {
        int pinLength = 4;
        int faultCode = 255;
        int triesRemaining;

        short count = setIncomingAndReceive();    // get expected data

        if(count < pinLength) throw new Exception();

        if(isInvalid() != false)
        {
            triesRemaining = getTriesRemaining();
            throw new Exception();
        }
    }


    private boolean isInvalid()
    {
        return true;
    }

    private short setIncomingAndReceive()
    {
        return 5;
    }

    private int getTriesRemaining()
    {
        return 2;
    }
}
\end{lstlisting}

\newpage

\begin{lstlisting}[caption={Mocked Java example code from the Java Card samples with the call graph integrity countermeasure implemented},label={lst:exampleCGI}]
// ExampleCGI.java
public class ExampleCGI
{
    private static int callId;

    public static void main(String[] args) {
        try
        {
            callId = 1;
		    ExampleCGI hw = new ExampleCGI();

        if(!(callId == 2))
        {
            throw new Exception();
        }

        }catch (Exception ex)
        {

        }
    }

    public ExampleCGI() throws Exception
    {
        if(callId != 1)
        {
            throw new Exception();
        }

        callId = 2;

        processVerifyPIN();

        if(callId != 3)
        {
            throw new Exception();
        }

        callId = 2;
    }

    private void processVerifyPIN() throws Exception
    {
        if(callId != 2)
        {
            throw new Exception();
        }

        int pinLength = 4;
        int faultCode = 255;
        int triesRemaining;

        callId = 3;

        short count = setIncomingAndReceive();    // get expected data

        if(callId != 4)
        {
            throw new Exception();
        }

        if(count < pinLength) throw new Exception();

        callId = 4;

        if(isInvalid() != false)
        {
            if(callId != 5)
            {
                throw new Exception();
            }

            callId = 5;
            triesRemaining = getTriesRemaining();
            
            if(callId != 6)
            {
                throw new Exception();
            }

            throw new Exception();
        }

        callId = 2;
    }


    private boolean isInvalid() throws Exception
    {
        if(callId != 4)
        {
            throw new Exception();
        }

        callId = 5;

        return true;
    }

    private short setIncomingAndReceive() throws Exception
    {
        if(callId != 3)
        {
            throw new Exception();
        }

        callId = 4;
        return 5;
    }

    private int getTriesRemaining() throws Exception
    {
        if(callId != 5)
        {
            throw new Exception();
        }

        callId = 6;

        return 2;
    }
}
\end{lstlisting}

\begin{lstlisting}[caption={Java bytecode example of the code duplication countermeasure},label={lst:exampleBytecode}]
Class Example

private bool  isInvalid ();  
Concrete Method
Parsed     

Example.processVerifyPIN()    
0.  iconst 1
1.  ireturn

 public  Example ();  Concrete Method   Parsed      Example.main(java.lang.String[])    
0.  aload  0
1.  invokespecial  void  java.lang.Object.<init> ()
4.  aload  0
5.  invokespecial  void  Example.processVerifyPIN ()
8.  return

 private void  processVerifyPIN ();  Concrete Method   Parsed      Example.<init>()    
0.  iconst 4
1.  istore  1
2.  sipush 255
5.  istore  2
6.  aload  0
7.  invokespecial  short  Example.setIncomingAndReceive ()
10.  istore  4
12.  iload  4
14.  iload  1
15.  ifcmpge 11
18.  new  java.lang.Exception
21.  dup
22.  invokespecial  void  java.lang.Exception.<init> ()
25.  athrow
26.  aload  0
27.  invokespecial  bool  Example.isInvalid ()
30.  ifeq 16
33.  aload  0
34.  invokespecial  int  Example.getTriesRemaining ()
37.  istore  3
38.  new  java.lang.Exception
41.  dup
42.  invokespecial  void  java.lang.Exception.<init> ()
45.  athrow
46.  aload  0
47.  invokespecial  bool  Example.isInvalid ()
50.  ifeq 4
53.  goto -20
54.  return

 public static void  main ( java.lang.String [] 0);  Concrete Method   Parsed  
0.  new  Example
3.  dup
4.  invokespecial  void  Example.<init> ()
7.  astore  1
8.  goto 4
11.  astore  1
12.  return

try start: 0; try end: 8: catch start: 11; catched type: java.lang.Exception.

private int  getTriesRemaining (); 
Concrete Method   
Parsed      

Example.processVerifyPIN()    
0.  iconst 2
1.  ireturn

private short  setIncomingAndReceive ();  Concrete Method   Parsed      Example.processVerifyPIN()    
0.  iconst 5
1.  ireturn

\end{lstlisting}

\begin{lstlisting}[caption={Java code example of the control flow integrity countermeasure},label={lst:examplecfi}]

public class ExampleCFI
{
	private static int flag = 0;
	
    public static void main(String[] args) {
        try
        {
		  ExampleCFI hw = new ExampleCFI();
        }catch (Exception ex)
        {

        }
    }

    public ExampleCFI() throws Exception
    {
        processVerifyPIN();
		
		if(flag != 3)
		{
			throw new Exception();
		}
    }

    private void processVerifyPIN() throws Exception
    {
		flag++;
		
        int pinLength = 4;
        int faultCode = 255;
        int triesRemaining;

        short count = setIncomingAndReceive();    // get expected data
		
        if(count < pinLength) throw new Exception();

        if(isInvalid() != false)
        {
            triesRemaining = getTriesRemaining();
            throw new Exception();
        }
    }


    private boolean isInvalid()
    {
		flag++;
        return true;
    }

    private short setIncomingAndReceive()
    {
		flag++;
        return 5;
    }

    private int getTriesRemaining()
    {
        return 2;
    }


}
\end{lstlisting}
\ch{fix indents in code}