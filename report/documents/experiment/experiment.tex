\chapter{Experiments}
In this chapter, experimental results are presented to confirm that implementing code duplication and call graph integrity countermeasures, \cref{sub:faultCounter}, makes Java code more and not less secure. The experiments will also be performed on code which has no countermeasures implemented from the \jc samples found in \cref{chap:samples}.\ch{explain why we chose flip in ifeq and why it might be safer to use ifneq (goto)}.
%, shown in \cref{lst:exUnmod}. 
%A bit flip inverts the comparison and the code in \cref{lst:exMod} is the result. The countermeasures are implemented on this, modified, code.\ch{}
%\begin{lstlisting}[label={lst:exUnmod}, caption=Purse code sample from the \jc samples.]
%if (USERPIN.isValidated)
%{
%}
%\end{lstlisting}
%
%\begin{lstlisting}[label={lst:exMod}, caption=Purse code sample from the \jc samples with a bit flipped to change the comparison.]
%if (!USERPIN.isValidated)
%{
%}
%\end{lstlisting}
\section{Setup}
The experiments will be tested according to two criteria, to determine whether a fault affecting security has happened

\begin{itemize}
\item The simulation can reach the main template's \textit{done} location without an exception occuring
\item The simulation can reach the main template's \textit{done} location without a operand stack fault happening
\end{itemize}

\section{Code Duplication}
% cd + pc
For CD + PC we expected a change in the attack probability compared to Base + PC, since the critical region of the code was offset, which might or might not have enabled new paths to be taken. The non-significant change in the results may be because the protected program happens to have the same amount of valid paths.\\\\
% cd + op
CD + OP successfully protects the code when it is subjected to a bit flip in the operand stack, as seen in the attack column. We attribute this to the fact that code that uses the operand stack, e.g. an \texttt{ifeq} instruction, is duplicated and therefore a flip in a value used, is overwritten with a correct value.
\section{Call Graph Integrity}
% cgi + pc
CGI + PC shows a higher vulnerability with a fault in the program counter, compared to the code duplication countermeasure. This is likely because the sensitive code region has become larger, as a result of additional instructions inserted to implement the countermeasure, thus resulting in a larger attack surface.\\\\
% cgi + op
As CGI + OP shows, a successful attack is possible, but it can not differentiated from Base + OP, as the probability for a successful attack is very small.
%Opstack pointer assumption is broken\\\\
%Inst fault og Instruction Differentiation\\\\
%Code dub\\\\
%Call graph integrity\\\\
