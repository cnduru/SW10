\chapter{Experiments}
In this chapter, experimental results are presented to confirm that implementing code duplication and call graph integrity countermeasures, see \cref{sub:faultCounter}, makes Java code more and not less secure. The experiments will also be performed on code which has no countermeasures implemented from the \jc samples found in \cref{chap:samples}.
%, shown in \cref{lst:exUnmod}. 
%A bit flip inverts the comparison and the code in \cref{lst:exMod} is the result. The countermeasures are implemented on this, modified, code.\ch{}
%\begin{lstlisting}[label={lst:exUnmod}, caption=Purse code sample from the \jc samples.]
%if (USERPIN.isValidated)
%{
%}
%\end{lstlisting}
%
%\begin{lstlisting}[label={lst:exMod}, caption=Purse code sample from the \jc samples with a bit flipped to change the comparison.]
%if (!USERPIN.isValidated)
%{
%}
%\end{lstlisting}
\section{\jc Purse}
The experiments are performed on a code sample from a selected part of the \jc samples \cref{lst:example}, where some parts, mainly variables and methods have been mocked since it was not necessary to model the complete samples.\\\\
In order to determine whether the implemented countermeasures do indeed provide improved protection against bit flips, compared to unprotected code, experiments are needed. These are performed with the modelling tool UPPAAL, which is described in \cref{chap:upp}.\\\\
The experiments will be run according to two criteria, to determine whether a fault affecting security has happened. Below, these criteria and their related UPPAAL queries are listed

\begin{itemize}
\item The simulation can reach the main template's \textit{done} location without an exception occuring
	\begin{itemize}
	\item \texttt{Pr[<= 100] (<> done \&\& !exceptionOccurred)}
	\end{itemize}
\item The simulation can reach the main template's \textit{done} location without a operand stack fault happening
	\begin{itemize}
	\item
	\end{itemize}
\end{itemize}

\ch{insert figure}
\ch{explain why we chose the same amount of time units for every program, even though they have different run-times}
We assume that a bit flip will occur within $80$ time units after program start. The bit flip may or may not have an effect, depending on whether it occurs between program start and program end, or outside.
\subsection*{Results}
The results in the table are listed in the following format in the extreme left column: $Code\:version + (attack)$, where content inside the parenthesis is optional.
\begin{table}[H]
\resizebox{\textwidth}{!}{
    \begin{tabular}{l|l|l|l|l|l|l}
    Version & No change & No change \% & Crash & Crash \% & Attack & Attack \% \\ \hline
    Base      & True &\relax[0.990, 1] & False & \relax[0, 0.01] & False & \relax[0, 0.01] ~    \\%todo
    Base + PC & False & \relax[0.854, 0.864] & True & \relax[0.108, 0.118] & True & [0.024, 0.034]    \\%todo
    Base + OP & False & [0.990, 1] & False & [0, 0.01] & True & [0, 0.01]    \\%todo
    Base + H  & True &\relax[0.990, 1] & False & \relax[0, 0.01] & False & \relax[0, 0.01]    \\%todo
    Base + L  & True &\relax[0.990, 1] & False & \relax[0, 0.01] & False & \relax[0, 0.01]    \\%todo
    CD        & True &\relax[0.990, 1] & False & \relax[0, 0.01] & False & \relax[0, 0.01]    \\%todo
    CD + PC   & False & \relax[0.853, 0.863] & True & \relax[0.108, 0.118] & True & [0.024, 0.034]     \\%todo
    CD + OP   & True &\relax[0.990, 1] & False & \relax[0, 0.01] & False & \relax[0, 0.01]   \\%todo
    CGI       & True &\relax[0.990, 1] & False & \relax[0, 0.01] & False & \relax[0, 0.01]   \\%todo
    CGI + PC  & False & \relax[0.723, 0.733] & True & \relax[0.227, 0.237] & True & \relax[0.031, 0.041]   \\%todo
    CGI + OP  & False & \relax[0.999, 1] & False & \relax[0, 0.001] & True & \relax[0.0002, 0.0012]     \\
    CFI       & True &\relax[0.999, 1] & False & \relax[0, 0.001] & False & \relax[0,0.001]    \\
    CFI + PC  & False & \relax[0.999, 1] & True & \relax[0.151, 0.152] & True & \relax[0.0286, 0.0296]   \\ %todo
    CFI + OP  & False & \relax[0.999, 1] & False & \relax[0, 0.001] & True & \relax[0.0001,0.0011]     \\
    CFI2       & True &\relax[0.999, 1] & False & \relax[0, 0.001] & False & \relax[0, 0.001]   \\
    CFI2 + PC  & False & \relax[0.825, 0.826] & True & \relax[0.150, 0.151] & True & \relax[0.024, 0.025]   \\ %todo
    CFI2 + OP  & False & \relax[0.999, 1] & False & \relax[0, 0.001] & True & \relax[2.266e-5, 0.001]      \\
    \end{tabular}}
    \caption{All experiment results have a confidence of $0.995$}
\end{table}

\noindent We do not utilise values from the heap or local variables to determine program flow, and as a result heap and local faults have no potential for attack. They are therefore omitted from the following experiment conclusions and the results table, except in the Base. The base case for each countermeasure are provided to show, that the countermeasure does not change the programs external behaviour.
\subsection{Base}
% base
In the Base case, where no fault was introduced, the progam ran as expected.\\\\
% base + pc
In the case of Base + PC, there is a change in the crash and attack probabilities. The changes are likely due to two cases: When the operand stack becomes unaligned, e.g. because the program counter is changed to an address containing an instruction which consumes two stack elements, but the original instruction only consumes one. Or when a bit flip changes the program counter to an invalid value.\\\\
% base + op
The results for Base + OP show that an attack is possible, however the probability simulation is not able to distinguish it from the Base case. This could be due to the fact, that the simulation did not enconter a run where an attack was possible.\\\\
\subsection{Code Duplication}
% cd + pc
For CD + PC we expected a change in the attack probability compared to Base + PC, since the critical region of the code was offset, which might or might not have enabled new paths to be taken. The non-significant change in the results may be because the protected program happens to have the same amount of valid paths.\\\\
% cd + op
CD + OP successfully protects the code when it is subjected to a bit flip in the operand stack, as seen in the attack column. We attribute this to the fact that code that uses the operand stack, e.g. an \texttt{ifeq} instruction, is duplicated and therefore a flip in a value used, is overwritten with a correct value.
\subsection{Call Graph Integrity}
% cgi + pc
CGI + PC shows a higher vulnerability with a fault in the program counter, compared to the code duplication countermeasure. This is because the sensitive code region has become larger, as a result of additional instructions inserted to implement the countermeasure, thus resulting in a larger attack surface.\\\\
% cgi + op
As CGI + OP shows, though low at $\leq 1\%$, a successful attack is possible in CGI + OP, while it is not in CD + OP. The reason for this, is that the chance of a bit flip in the operand stack occurring at exactly the right time, at the right place in the operand stack is very small.
\subsection{Summary}
% code dup
The experiments show that code duplication protects the particular code sample better than call graph integrity. This is because none of the faults introduced, alter the call graph itself, they only change the program flow from one path to an already existing path.\\\\
% inst fault parameter
A fault model we did not include is instruction parameter faults, such as flipping a bit in the target address of a \texttt{goto} or method index in an invoke. This could cause a change in the call graph. Additionally, if the examples had used virtual methods, there would be a chance of calling a method based on the wrong class id, caused by a bit flip. Call graph integrity would catch both of these cases.\\\\
% CGI
%CFI & CFI2


% virtual

\begin{tabular}{|c|c|c|c|c|}
\hline Version & Cd & CF1 & CFI2 & CGI \\ 
\hline Purse &  &  &  &  \\ 
\hline Invoke &  &  &  &  \\ 
\hline 
\end{tabular} 

% % % % %

\section{INVOKE Sample}
The experiments are performed on the code sample from \cref{lst:virtual}, results for the code duplication countermeasure are omitted because the example has no branches. The purpose of the sample is to include virtual and special invokes, to be used in these experiments. Because of the invokes, heap and locals faults now have an impact, and are therefore included in the result table.\\\\
\ch{mention that exceptions are not caught so they show themselves as crashes in the results}
\subsection*{Results}
\begin{table}[H]
\resizebox{\textwidth}{!}{
    \begin{tabular}{l|l|l|l|l|l|l}
    Version & No change & No change \% & Crash & Crash \% & Attack & Attack \% \\ \hline
    Base      & True  & \relax[0.990, 1] & False & \relax[0, 0.01] & False & \relax[0, 0.01]    \\
    Base + PC & False & \relax[0.385, 0.395] & True & \relax[0.607, 0.617] & True & [0.023, 0.032]  \\
    Base + OP & False & \relax[0.961, 0.971] & True & [0.005, 0.015] & True & [0.013, 0.023] \\
    Base + H  & False & \relax[0.912, 0.922] & True & \relax[0.053, 0.063] & True & \relax[0.004, 0.014] \\
    Base + L  & False & \relax[0.966, 0.976] & True & \relax[0.014, 0.024] & True & \relax[0.004, 0.014] \\
    CGI       & True  & \relax[0.990, 1] & False & \relax[0, 0.01] & False & \relax[0, 0.01]    \\
    CGI + PC  & False & \relax[0.032, 0.042] & True & \relax[0.859, 0.869] & True & [0.041, 0.051]    \\
    CGI + OP  & False & \relax[0.928, 0.938] & True & [0.005, 0.015] & True & [0.046, 0.056]    \\
    CGI + H   & False & \relax[0.821, 0.831] & True & \relax[0,111 0.121] & True & \relax[0.024, 0.034]    \\
    CGI + L   & False & \relax[0.937, 0.947] & True & \relax[0.027, 0.0037] & True & \relax[0.012, 0.022]    \\
    CFI       & True  & \relax[0.990, 1] & False & \relax[0, 0.01] & False & \relax[0, 0.01]    \\
    CFI + PC  & False & \relax[0.147, 0.157] & True & \relax[0.851, 0.861] & True & [0.007, 0.017]    \\
    CFI + OP  & False & \relax[0.938, 0.948] & True & [0.030, 0.040] & True & [0.016, 0.026]    \\
    CFI + H   & False & \relax[0.829, 0.839] & True & \relax[0.131, 0.141] & True & \relax[0.008, 0.018]    \\
    CFI + L   & False & \relax[0.964, 0.974] & True & \relax[0.017, 0.27] & True & \relax[0.001, 0.011]    \\

    \end{tabular}}
    \label{tap:invoke}
    \caption{Experiment results for invoke example. All experiment results have a confidence of $0.995$}
\end{table}
\subsection{Base}
% base
In the base case, no fault was introduced and as a result, the program ran as expected.\\\\
%
% base + pc
In Base + PC, the crash probability appear high which is likely because of misaligned program counter values.\\\\
%
% base + op
The probability of crashes in Base + OP are present because the sample uses the operand stack to a greater degree than the \jc Purse results, thus causing a larger attack surface.
%
% base + H + L
Results for Base + H and Base + L show detectable crash and attack probabilities, because the sample uses the heap for objects and local variables for storing calculation results.
\subsection{Code Duplication}
\input{documents/experiment/codedupVirtual}
\subsection{Call Graph Integrity}
Generally, the CGI countermeasure did not work well on the code sample, and the probability of a successful attack appears to be greater than for the Base example. We attribute this to the fact that the CGI code size was twice as big as the Base, and as a consequence created a larger attack surface, while CFI added significantly less size overhead.
\subsection{Summary}
\input{documents/experiment/summaryVirtual}

%\subsection{Instruction fault}
%\ch{explain why we chose flip in ifeq and why it might be safer to use %ifneq (goto)}.
%\kri{maybe move this as no experiments will be done on it}
%Opstack pointer assumption is broken\\\\
%Inst fault og Instruction Differentiation\\\\
%Code dub\\\\
%Call graph integrity\\\\
