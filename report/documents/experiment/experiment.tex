\chapter{Experiments}
In this chapter, experimental results are presented to confirm that implementing code duplication and call graph integrity countermeasures, see \cref{sub:faultCounter}, makes Java code more and not less secure. The experiments will also be performed on code which has no countermeasures implemented from the \jc samples found in \cref{chap:samples}.\\\\
In order to determine whether the implemented countermeasures do indeed provide improved protection against bit flips, compared to unprotected code, experiments are needed. These are performed with the modelling tool UPPAAL, which is described in \cref{chap:upp}.
%, shown in \cref{lst:exUnmod}. 
%A bit flip inverts the comparison and the code in \cref{lst:exMod} is the result. The countermeasures are implemented on this, modified, code.\ch{}
%\begin{lstlisting}[label={lst:exUnmod}, caption=Purse code sample from the \jc samples.]
%if (USERPIN.isValidated)
%{
%}
%\end{lstlisting}
%
%\begin{lstlisting}[label={lst:exMod}, caption=Purse code sample from the \jc samples with a bit flipped to change the comparison.]
%if (!USERPIN.isValidated)
%{
%}
%\end{lstlisting}
\section{\jc Purse}
The experiments will be tested according to two criteria, to determine whether a fault affecting security has happened

\begin{itemize}
\item The simulation can reach the main template's \textit{done} location without an exception occuring
\item The simulation can reach the main template's \textit{done} location without a operand stack fault happening
\end{itemize}

\subsection*{Results}
The results in the table are listed in the following format in the extreme left column: $Code\:version + (attack)$, where attack is optional.
\begin{table}[H]
\resizebox{\textwidth}{!}{
    \begin{tabular}{l|l|l|l|l|l|l}
    Version & No change & No change \% & Crash & Crash \% & Attack & Attack \% \\ \hline
    Base      & True &\relax[0.990, 1] & False & \relax[0, 0.01] & False & \relax[0, 0.01]    \\
    Base + PC & False & \relax[0.640, 0.650] & True & \relax[0.342, 0.352] & True & [0.004, 0.014]    \\
    Base + OP & False & [0.990, 1] & False & [4,4e-5, 0.01] & True & [0, 0.01]    \\
    Base + H  & True &\relax[0.990, 1] & False & \relax[0, 0.01] & False & \relax[0, 0.01]    \\
    Base + L  & True &\relax[0.990, 1] & False & \relax[0, 0.01] & False & \relax[0, 0.01]    \\
    CD        & True &\relax[0.990, 1] & False & \relax[0, 0.01] & False & \relax[0, 0.01]    \\
    CD + PC   & False & \relax[0.653, 0.663] & True & \relax[0.331, 0.341] & True & [0.003, 0.013]     \\
    CD + OP   & True &\relax[0.990, 1] & False & \relax[0, 0.01] & False & \relax[0, 0.01]   \\
    CGI       & True &\relax[0.990, 1] & False & \relax[0, 0.01] & False & \relax[0, 0.01]   \\
    CGI + PC  & False & \relax[0.277, 0.287] & True & \relax[0.701, 0.711] & True & \relax[0.012, 0.022]   \\
    CGI + OP  & False & \relax[0.990, 1] & False & \relax[0, 0.01] & True & \relax[0.01, 0.01]     \\
    CFI       & True &\relax[0.990, 1] & False & \relax[0, 0.01] & False & \relax[0,0.01]    \\
    CFI + PC  & False & \relax[0.504, 0.514] & True & \relax[0.478, 0.488] & True & \relax[0.005, 0.015]   \\
    CFI + OP  & False & \relax[0.990, 1] & False & \relax[0, 0.01] & True & \relax[0,0.01]     \\
    CFI2       & True &\relax[0.990, 1] & False & \relax[0, 0.01] & False & \relax[0, 0.01]   \\
    CFI2 + PC  & False & \relax[0.558, 0.568] & True & \relax[0.428, 0.438] & True & \relax[0.004, 0.014]   \\
    CFI2 + OP  & False & \relax[0.990, 1] & False & \relax[0, 0.01] & True & \relax[2.266e-5, 0.01]      \\
    \end{tabular}}
    \caption{Experiment results for \jc Purse example. All experiment results have a confidence of $0.995$}
\end{table}

\noindent We do not utilise values from the heap or local variables to determine program flow, and as a result heap and local faults have no potential for attack. They are therefore omitted from the following experiment conclusions and results table, except in the Base version of the code sample. The base case for each countermeasure is provided to show that the countermeasure does not change the program's external behaviour.

\kri{make sure we don't repeat in the following}

\subsubsection{Base}
% base
In the Base case, where no fault was introduced, the progam ran as expected.\\\\
% base + pc
In the case of Base + PC, interesting results emerge. In $10.8\% - 11.8\%$ of runs, a crash occurred. This happens when the operand stack becomes unaligned, e.g. because the program counter is changed to an address containing an instruction which consumes two stack elements, but the original instruction only consumes one.\\\\
% base + op
The results for Base + OP show that it is possible to perform a successful attack, but the probability is small at $\leq 1\%$.\\\\
\subsubsection{Code Duplication}
% cd + pc
For CD + PC we expected a change in the attack probability compared to Base + PC, since the critical region of the code was offset, which might or might not have enabled new paths to be taken. The non-significant change in the results may be because the protected program happens to have the same amount of valid paths.\\\\
% cd + op
CD + OP successfully protects the code when it is subjected to a bit flip in the operand stack, as seen in the attack column. We attribute this to the fact that code that uses the operand stack, e.g. an \texttt{ifeq} instruction, is duplicated and therefore a flip in a value used, is overwritten with a correct value.
\subsubsection{Call Graph Integrity}
% cgi + pc
CGI + PC shows a higher vulnerability with a fault in the program counter, compared to the code duplication countermeasure. This is likely because the sensitive code region has become larger, as a result of additional instructions inserted to implement the countermeasure, thus resulting in a larger attack surface.\\\\
% cgi + op
As CGI + OP shows, a successful attack is possible, but it can not differentiated from Base + OP, as the probability for a successful attack is very small.
\subsubsection{Call Flow Integrity}
Similar to CGI + PC, CFI + PC also has an increased crash probability, but without an increased attack probability compared to Base. 
It was expected that CFI might reduce the probability of a pc fault attack, as it introduces checks to confirm that important code has been run. 
\kri{make sure this is understandable}
We suspect the reason for this is the structure of the code, as there is no sensitive region where flags can be rechecked and bypassing the methods doing the validation is enough to be considered a successful attack.

CFI + OP shows no differences compared to base + OP. As expected, control flow integrity does not protect against errors in the operand stack. 
\subsubsection{Call Flow Integrity 2}
CFI2 is a modification of the CFI example seen in \cref{lst:examplecfi}, which was our attempt\ch{replace authors with we} to reduce the attack probability.\\\\
% CFI2 + PC
As the results show in the case of CFI2 + PC, the modification is hard to separate from the attack probability of CFI + PC, and it had no discernible effect. A positive side effect, however, was a reduced crash probability, which is likely due to the fact that the largest method in CFI2 has fewer instructions than in CFI, and thus a smaller attack surface.
% CFI2 + OP

\subsubsection{Summary}
\documentclass{article}
\usepackage{graphicx}

\begin{document}

\title{Summary}
\author{Kristian M. Thomsen and Christoffer Nd\~ur\~u}
\maketitle

\section{Summary}
% bite
In May 2011 the Economic Interest Group discovered that smart cards stolen in France were being used for transactions in Belgium. It turned out a group of criminals had managed to bypass the PIN verification on the cards and could use them for purchasing items, which they would later sell on the black market. Since smart cards are a widespread technology, for example in credit cards, abuse of them poses serious risks to both the banking industry, but also to consumers.\\\\
This report presents fault injections on the Java Card platform. It is based on previous work formalising a subset of Java Card bytecode and fault models. The architecture of the Java Card platform is presented and how persistent and transient faults can affect it.\\\\
% %
An approach to automatic conversion of Java bytecode to UPPAAL models is also detailed. In extension, approaches for automatic modelling of a variety of fault injection attacks are described.
% %
Several known fault injection countermeasures are also presented, accompanied by a solution to automate model based safety analysis of Java Card programs, by inserting attacks into code modified with countermeasures, and modelling them in the modelling tool UPPAAL.\\\\
% %
The solution uses the tool $Sawja$ to provide a convenient representation of Java bytecode, which makes bytecode more readable. The tool is able to produce call graphs of programs, which can be used for automation of countermeasures. Improvements to the solution are also offered to allow future work, including suggestions for the first steps in automating implementation of control flow and control graph integrity countermeasures, which could be used to create a solution that can compare countermeasures' abilities to protect against fault injections.\\\\
% %
A series of experiments are also conducted in an attempt to compare countermeasures' protection level against two selected code bases. In extension, we explore the viability of our experiment approach used to compare countermeasures.

\end{document}

% % % % %

\section{Invoke}
The experiments are performed on the code sample from \cref{lst:virtual}, results for the code duplication countermeasure are omitted because the example has no branches. The purpose of the sample is to include virtual and special invokes, to be used in these experiments. Because of the invokes, heap and locals faults now have an impact, and are therefore included in the result table.\\\\
\subsection*{Results}
\begin{table}[H]
\resizebox{\textwidth}{!}{
    \begin{tabular}{l|l|l|l|l|l|l}
    Version & No change & No change \% & Crash & Crash \% & Attack & Attack \% \\ \hline
    Base      & True  & \relax[0.990, 1] & False & \relax[0, 0.01] & False & \relax[0, 0.01]    \\
    Base + PC & False & \relax[0.385, 0.395] & True & \relax[0.607, 0.617] & True & [0.023, 0.032]  \\
    Base + OP & False & \relax[0.961, 0.971] & True & [0.005, 0.015] & True & [0.013, 0.023] \\
    Base + H  & False & \relax[0.912, 0.922] & True & \relax[0.053, 0.063] & True & \relax[0.004, 0.014] \\
    Base + L  & False & \relax[0.966, 0.976] & True & \relax[0.014, 0.024] & True & \relax[0.004, 0.014] \\
    CGI       & True  & \relax[0.990, 1] & False & \relax[0, 0.01] & False & \relax[0, 0.01]    \\%todo
    CGI + PC  & False & \relax[0.640, 0.650] & True & \relax[0.342, 0.352] & True & [0.004, 0.014]    \\%todo
    CGI + OP  & False & \relax[0.990, 1] & False & [4,4e-5, 0.001] & True ue& [0, 0.001]    \\%todo
    CGI + H   & True  & \relax[0.990, 1] & False & \relax[0, 0.001] & False & \relax[0, 0.001]    \\%todo
    CGI + L   & True  & \relax[0.990, 1] & False & \relax[0, 0.001] & False & \relax[0, 0.001]    \\%todo
    CFI       & True  & \relax[0.990, 1] & False & \relax[0, 0.01] & False & \relax[0, 0.01]    \\
    CFI + PC  & False & \relax[0.208, 0.218] & True & \relax[0.849, 0.859] & True & [0.024, 0.034]    \\
    CFI + OP  & False & \relax[0.956, 0.966] & True & [0.031, 0.041] & True & [0.014, 0.024]    \\
    CFI + H   & False & \relax[0.886, 0.896] & True & \relax[0.131, 0.141] & True & \relax[0.008, 0.018]    \\
    CFI + L   & False & \relax[0.962, 0.972] & True & \relax[0.015, 0.025] & True & \relax[0.005, 0.015]    \\

    \end{tabular}}
    \caption{Experiment results for invoke example. All experiment results have a confidence of $0.995$}
\end{table}
\subsubsection{Base}
% base
In the base case, no fault was introduced and as a result, the program ran as expected.\\\\
%
% base + pc
In Base + PC, the crash probability appear high which is likely because of misaligned program counter values.\ch{replace pc with program counter in earlier descriptions}~\\\\
%
% base + op
The probability of crashes in Base + OP are present because the sample uses the operand stack to a greater degree than the \jc Purse results, thus causing a larger attack surface.
%
% base + H + L
Results for Base + H and Base + L show detectable crash and attack probabilities, because the sample uses the heap for objects and local variables for storing calculation results.
\subsubsection{Code Duplication}
test
\subsubsection{Call Graph Integrity}
Generally, the CGI countermeasure did not work well on the code sample, and the probability of a successful attack appears to be greater than for the Base example. We attribute this to the fact that the CGI code size was twice as big as the Base, and as a consequence created a larger attack surface, while CFI added significantly less size overhead.
\subsubsection{Summary}
test

%\subsection{Instruction fault}
%\ch{explain why we chose flip in ifeq and why it might be safer to use %ifneq (goto)}.
%\kri{maybe move this as no experiments will be done on it}
%Opstack pointer assumption is broken\\\\
%Inst fault og Instruction Differentiation\\\\
%Code dub\\\\
%Call graph integrity\\\\
