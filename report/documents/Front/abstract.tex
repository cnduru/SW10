% emnet
Attacks against smart cards are getting ever more advanced and it is thus important to stay ahead on the security front. 
Many different countermeasures exist to counter the attacks, but it can be difficult to choose appropriate countermeasures to protect a particular program. 
The purpose of this report is to address this concern by analysing how bit flips, induced by fault attacks, can affect Java programs. Previous work defining formal semantics and fault models for a small Java Card inspired language, are extended to provide a tool for automatic conversion of Java bytecode to UPPAAL models. We use UPPAAL to verify security properties with respect to formalised fault models.

%%% Local Variables:
%%% mode: plain-tex
%%% TeX-master: "../../master"
%%% End:

%SW7 ABSTRACT
%The purpose of this project is to develop a system that is able to retrieve and distribute news articles crawled from external news providers such as NY Times and BBC.
%It can thus be characterized as a content curator, as it is able to retrieve, manipulate and present the crawled articles to users in the context of a web application.
%Content curation is enabled by implementing a recommender system that is able to suggest recommended articles to users by using collaborative filtering, while also considering the cold-start problem when no information about the user is know.
%Recommendations are computed in context of like-minded users and interests extracted from a social network profile.
%Retrieval is achieved through a crawler that observes a selected number of RSS feeds from many different providers.
%Whenever a new article is found from a feed, its content is manipulated through a series of steps that e.g. extracting content relevant to the domain and also classifying the content using text classification.
%After manipulating an article it is stored to enable a web application to distribute the articles to users.
