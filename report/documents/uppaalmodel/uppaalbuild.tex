\chapter{Building a UPPAAL Model}
\section{Choosing a Program Representation}
When translating a program to a UPPAAL of the program. Several representations are possible, depending on what one wants to show. One could for example represent a program merely in terms of program flow if a simulation of a disruption of the program flow is to be shown, e.g an error in the program counter. One could also include the data flow in the program if a simulation of a corruption of a memory value is to be shown. These are just a few examples and many representations can be chosen.\\

We have chosen the later and model the program in terms of program flow and data flow, so that we can simulation disruptions in the programs execution flow. 


\subsection{Representation Details}
The program simulation is based on \jcl semantics, most Java bytecodes can be translated directly to \jcl at the loss of type information.\\

When representing Java bytecode in UPPAAL we have chosen to represent an instruction, such as \texttt{iload a} and \texttt{dub}, as UPPAAL locations. This implies that a change in the program counter is a change of the location. In turn this means that when an instruction is to be executed the change to the program configuration \textit{Conf} \cref{sec:semintro} \kristian{ref til semantik}.

\section{Rewriting the Model}