\section{Program Modelling}
When translating a program to a UPPAAL model, several approaches are possible, depending on what is to be shown. One could for example represent a program merely in terms of program flow, if a disruption of program flow is to be simulated, e.g an error in the program counter. A memory corruption could be simulated by including the data flow in the model. These are just a few examples and many representations can be chosen. We have chosen the latter and model programs in terms of program and data flow to simulate disruptions in the execution flow.
%
\subsection{Program Simulation}
\kri{simulation or modelling}
The program simulation is based on \jcl semantics, most Java bytecodes can be translated directly to \jcl at the loss of type information.\\
\kri{what does it mean for us?}
When representing Java bytecode in UPPAAL we have chosen to represent an instruction, such as \texttt{aload a} and \texttt{dup}, as UPPAAL locations. 
This implies that a change in the program counter is a change of the location. 
In turn this means that when an instruction is to be executed the change to the program configuration \textit{Conf} from  \cref{sec:semintro} occurs on the edge to next location.

\begin{minipage}{\linewidth}
\begin{lstlisting}[caption=Java code sample.]
public class Sample{
    public static void main(String[] args) {
        for (String a : args)         
        {
            System.out.print(a);
        }
    }
}
\end{lstlisting}
\end{minipage}
\begin{minipage}{\linewidth}
\begin{lstlisting}[caption=Bytecode sample.]
public static void main(java.lang.String[]);
 Code:
   0: aload_0       
   1: astore_1      
   2: aload_1       
   3: arraylength   
   4: istore_2      
   5: iconst_0      
   6: istore_3      
   7: iload_3       
   8: iload_2       
   9: if_icmpge     31
  12: aload_1       
  13: iload_3       
  14: aaload        
  15: astore        4
  17: getstatic     #2                  // Field java/lang/System.out:Ljava/io/PrintStream;
  20: aload         4
  22: invokevirtual #3                  // Method java/io/PrintStream.print:(Ljava/lang/String;)V
  25: iinc          3, 1
  28: goto          7
  31: return        

\end{lstlisting}
\end{minipage}

\begin{figure}[H]
\includegraphics[width=\textwidth]{generated_wip}
\caption{Auto generated model, work in progress.}
\label{fig:generated_wip}
\end{figure}
\kri{update the model for the new sample}

\subsubsection{Simple Instructions}
\begin{figure}[H]
\centering
\begin{subfigure}{.3\textwidth}
  \begin{lstlisting}
  0. aload 0
  1. arraylength
  ...
  \end{lstlisting}
  \caption{Java Bytecode Sample.}
\end{subfigure} 
\hspace{10px}
\begin{subfigure}{.6\textwidth}
  \includegraphics[width=\textwidth]{UPPAAL1.png}
  \caption{Generated model from Sample.}
\end{subfigure}
\caption{Java bytecode and corresponding UPPAAL model.}
\label{fig:uppaal1}
\end{figure}
\kri{lav ny model til nyt sample}
\Cref{fig:uppaal1} show how two Java Bytecode instructions are represented in UPPAAL. On the left we see the Java bytecode, the first line with program counter 0 we have the \texttt{aload 0} instruction. \texttt{aload 0} pushes a reference to the top of the operation stack from local variables at position zero, then increments the operand stack pointer and program counter.\\\\
%uppaal location edge
In UPPAAL the location \texttt{pc0\_aload} represents \texttt{aload 0}. The UPPAAL model is seen in \Cref{fig:uppaal1}b. We simulate execution time with the location invariant \texttt{t <= 1}  and guard \texttt{t == 1} on the edge leading to the next location. The guard is found right below the location name right of the edge and invariant is to the left of the edge. In this sample we defined the execution time as 1 time unit.\\\\
In the update on the edge seen below the guard, we simulate the data flow by assigning the value of the local variable \texttt{loc0} to the top of the operand stack \texttt{os} represented by operand stack pointer \texttt{osp}. \texttt{osp} is incremented as the operand stack grows, the increment of the program counter is simulated by the edge itself.

\subsubsection{Jumps and Branches}
For the majority of instructions the program counter is set to the next instruction after execution, but for a jump with \texttt{goto a} the edge goes to the instruction with the program counter corresponding with value of \texttt{a}.

Conditionals such as \texttt{if\_cmpeq a} is the only instruction that is modelled by a location having two outgoing edges , one to the next instruction and one for the program counter of a. On these edges the guard is used to determine which of the edges is to be traversed. \kri{insert example}

\subsection{Method Calls}\label{subsubsec:method}
Method calls covers the Java bytecode instructions:~\texttt{invokestatic} for static calls, \texttt{invokespecial} for class constructors and private calls, and \texttt{invokevirtual} virtual calls. To illustrate how these instructions are modelled, we use the Java code sample in \cref{lst:virtual}.
\begin{lstlisting}[caption={\texttt{Bclass} extends \texttt{Aclass}, \texttt{Aclass} implements the methods foo and bar, and \texttt{Bclass} overwrites foo.}, label={lst:virtual}]
public class Virtual{
  public Aclass a;
  public Aclass b;

  public Virtual(){
    a = new Aclass();
    b = new Bclass();
    int ia = a.foo() + a.bar();
    int ib = b.foo() + b.bar();
  }
}
\end{lstlisting}
The sample includes the bytecode instructions \texttt{invokespecial} and \texttt{invokevirtual}.
\texttt{invokestatic} is omitted as \texttt{invokestatic} and \texttt{invokespecial} are nearly identical, with the only difference being if an object reference from the operand stack is parsed as a parameter.
As such all methods call can be divided into two categories, virtual and static.\\\\
These categories are defined in the Java specification: ``Virtual methods are instance methods that are resolved dynamically. The set includes all public, protected and package-visible instance methods. Private instance methods and all constructors are not virtual methods, but instead are resolved statically during compilation.'' \cite[chap. 4.3.7.6]{java_card_spec}

\subsubsection{Static Methods}
Static method calls are represented by three additions to the model. These additions consist of locations and edges.\\\\ 
%, but they do not have any associated program counter since they are not a part of the original program.\\ what?
% caller
The first is a new location in the caller for every method call it performs. This makes it possible to simulate parameter passing from the caller, as well as control transfer when waiting for a callee to return control after a method call. The simulation of the caller remains in this location until the callee returns control, after its simulation has finished. This control transfer is modelled with a synchronisation on the edge, going from the new state in the caller and back to its original control flow, as seen in \cref{fig:invokespecial} b.\\\\
% callee
The second is an addition of one additional location in every template. The first, initial location, \texttt{Aclass}, in \cref{fig:invokespecial} c, serves two purposes: it enables the control transfer from the caller to itself by synchronisation, and simulates passing of arguments into the method from the caller.\\\\
% main case and return
The third is the edge from the \texttt{return} instruction, seen in \cref{fig:invokespecial} c. This is one of the two edges pointing to the \texttt{AClass} initial location, and the other is for exceptions, which will be covered in \kri{ref to section}. For main, the edge goes to a \textit{Done} location instead of the initial location, where the simulation ends when it has finished. For other templates this is where control is transferred back to the caller, and the edge goes to the initial location.


\begin{figure}
\centering
\begin{subfigure}{\textwidth}
  \begin{lstlisting}
...
9. invokespecial void Aclass.<init> ( )
12. putfield Virtual.a : Aclass
...
  \end{lstlisting}
  \caption{Invokespecial Bytecode generated by Sawja.}
\end{subfigure} \\
%\hspace{10px}
\begin{subfigure}{.65\textwidth}
  \includegraphics[width=\textwidth]{invokespecial.png}
  \caption{Invokespecial Instruction.}
\end{subfigure}
\hspace{10px}
\begin{subfigure}{.25\textwidth}
  \includegraphics[width=\textwidth]{initloc.png}
  \caption{Initial Location.}
\end{subfigure}
\caption{Java bytecode and corresponding UPPAAL model.}
\label{fig:invokespecial}
\end{figure}


\subsubsection{Virtual Methods}
Virtual methods are similar to static methods in regards to representation in the method template for caller and callee, but instead of handling control directly to callee method templates, a template responsible for resolving the virtual call is inserted for this purpose.
\cref{fig:invokevirtual} is the resolver template generated for the code in \cref{lst:virtual}, there is a total of three virtual methods in this sample and the resolver has a waiting location for each.\\\\
Every class is mapped to an integer $clID$ and an array, $classHierarchy$, represents the class hierarchy of the program. The initial location \texttt{Invoke} waits for a synchronisation, after which the location \texttt{Resolver} has an outgoing edge for every possibility, in this sample that is five. There are essentially always three distinct possibilities 

\begin{itemize}
\item There are no methods and no super where $clID == 0$.
\item There is a method matching the class and method signature. In this case, call the method.
\item There are no methods but there is a super class, then assign $clID = classHierarchy[clID]$ and try again.
\end{itemize}

This is based on \texttt{methodLookup} from the \texttt{invokevirtual} semantics \cref{app:invokevirtual}.

 
\begin{figure}[H]
\centering
\includegraphics[width=\textwidth]{invokevirtual.png}
\caption{Invokevirtual.}
\label{fig:invokevirtual}
\end{figure}




\subsection{Fault Modelling}
We focus on three faults: program counter, data fault and static instruction fault where program counter and data fault is considered transient faults and static instruction fault is considered a persistent fault as described in \cref{sec:faultsce}.

\subsubsection{Program Counter Fault}

To model a single bit flip occurring in the program's execution a special fault template is introduced, illustrated in \cref{fig:faultTime}. The template selects a random value between $0$ and the maximum possible global clock value, which represents when in the programs execution a fault happens. The random value is assigned to a global variable in the UPPAAL system.\\\\
Every instruction in the Java bytecode is represented by a location, and has an associated program counter. There are edges from each location going to the locations which can be reached if one bit is flipped in the program counter. These edges have guards which check whether the time the fault is injected, corresponds to the global clock at the time the model simulation is at that particular edge. If it is, the guard will allow the edge to be traversed. There are no fault edges going back to the added locations described in \cref{subsubsec:method}, since these are not a part of the original program and therefore do not have an associated program counter.
\begin{figure}[H]
\centering
\includegraphics[width=0.3\textwidth]{figures/faulttemp.PNG}
\caption{The UPPAAL template which selects when to perform a bit flip in the program counter}
\label{fig:faultTime}
\end{figure}
\ch{update figure to use fault at between 0 and global clock}

\subsubsection{Data Faults}
% locals and opstack
To introduce data fault for the operand stack a special template selects when a fault should be introduced into the operand stack. 
This happens in the same way as in the \textit{program counter} fault described earlier, illustrated in \cref{fig:faultTime}. 
The selected value is between $0$ and the maximum possible runtime of the program.
\cref{fig:opstackFlip} shows how a method called \texttt{getTriesRemaining}, in which a bit flip in the operand stack occurs.
Edges going back to the locations \texttt{pc0\_iconst\_2} and \texttt{pc0\_iconst\_2} are where the faults are introduced, these are added by the solution and are not a part of an unaltered model of a program. 
The edges have a guard, $faultTime \leq globalClock$, determined by the special template, which only allows a fault to happen if the program execution has executed for a certain amount of time. 
The fault itself is introduced by the update $os\lbrack osPos \rbrack\:\hat{}= 1 \ll osBitPos$, which flips bit \UPP{osBitPos} of value \UPP{osPos} in the operand stack.
\UPP{osBitPos} is a random value between $0$ and $7$, which denotes which bit should be flipp
ed. \UPP{osPos} is a random value between $0$ and the maximum size of the operand stack. 
After a fault has been introduced, the variable \UPP{faultTime} is set to a value higher than the maximum value of the global clock, to ensure only one fault happens per simulation.\\\\
Our approaches to modelling faults in the operand stack, heap and local variables are similar and only one of them is therefore described.
% 	% % % %
\begin{figure}[H]
\centering
\includegraphics[width=\textwidth]{getTriesRemainnig.PNG}
\caption{The UPPAAL model af a method where a bit flip occurs in the operand stack.}
\label{fig:opstackFlip}
\end{figure}
\subsubsection{Instruction Fault}
\kri{can be both transient and presistent, update subsection}
Instruction faults are modelled by first assigning each edge of all templates a unique identifier. A special template then selects a random value in the range of $0$ and the greatest identifier in the modelled program. It is not time-dependent, i.e. it does not rely on a clock the \textit{operand stack} fault described earlier. A fault will only happen once since only one identifier is selected, and only when the simulation reaches the instruction chosen by the special template. This is enforced using guards which compares the selected identifier with the current edge's. During the generation of the model, the solution has calculated all instructions, an instruction can be changed to by a bit flip in their binary encoding. Additional edges are then inserted to perform the actions of the altered instructions. For example, the \texttt{ifeq} instruction, which can be bit flipped to \texttt{goto}, would cause an extra edge representing the "yes" branch to be created to the destination location of the original "yes" branch. The new edge is different from the two original edges by always being enabled, thus causing a simulation to always perform a jump regardless of the result of an \texttt{ifeq} comparison.
\section{Proposed Solution}
We propose a solution which can convert a Java class file to a UPPAAL SMC model, rewrite it to insert fault injection attack countermeasures and recommend one of these countermeasures. The point of the conversion to a model, is to be able to provide guarantees about certain properties of the program. This makes sense when the Java class file is rewritten and countermeasures are inserted, since it is then possible to guarantee that the program has not become less secure with respect to certain properties.\\

The workflow stages of the solution are illustrated in \cref{fig:workflow}. The stages are labelled with numbers 1-5. Their purposes are detailed in the following.

% burde vi ogs� give mulighed i programmet for bare at f� en ``vanilla'' SMC model ud?

\paragraph{Stage 1 - 2} rewrites the original Java class file to include fault injection countermeasures. The output is one or more modified Java class files depending on the number of countermeasures implemented. 
\paragraph{Stage 2 - 3} provides two options, either one can use the rewritten code to perform static analysis with an analysis tool, or one can parse the code through the solution's parser and generate a UPPAAL SMC model. 
\paragraph{Stage 3 - 4} modifies the model to include a special fault injection template, which can simulate a fault being injected in the Java program. At this point in the workflow, the model of the rewritten code is also modified to include special synchronisations, guards, updates and locations. These make it possible to verify timed properties of the model and enables the fault template to interact with the model.
\paragraph{Stage 3,4 - 5} is where the solution recommends a fitting countermeasure for the code, based properties it has verified about the rewritten version(s) of the code. These models serve as the input for stage 5. The recommended countermeasure incurs the least static size overhead of the rewritten Java class file but provides the most protection for the largest possible duration of the program's execution.


\begin{figure}[H]
\centering
\includegraphics[scale=0.9]{workflow.png}
\caption{The workflow of the solution, from Java class file to UPPAAL SMC model}
\label{fig:workflow}
\end{figure}
\ch{make better figure}
